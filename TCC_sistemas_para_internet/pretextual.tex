% Retira espaço extra obsoleto entre as frases.
\frenchspacing 

% \pretextual

% ---
% Capa
% ---
\imprimircapa
% ---

% ---
% Folha de rosto
% (o * indica que haverá a ficha bibliográfica)
% ---
%\imprimirfolhaderosto*
% ---

% ---
% Inserir a ficha bibliografica
% ---

% Isto é um exemplo de Ficha Catalográfica, ou ``Dados internacionais de
% catalogação-na-publicação''. Você pode utilizar este modelo como referência. 
% Porém, provavelmente a biblioteca da sua universidade lhe fornecerá um PDF
% com a ficha catalográfica definitiva após a defesa do trabalho. Quando estiver
% com o documento, salve-o como PDF no diretório do seu projeto e substitua todo
% o conteúdo de implementação deste arquivo pelo comando abaixo:
%
% \begin{fichacatalografica}
%     \includepdf{fig_ficha_catalografica.pdf}
% \end{fichacatalografica}
%\begin{fichacatalografica}
%	\hrule							% Linha horizontal
%	\begin{center}					% Minipage Centralizado
%	\begin{minipage}[c]{12.5cm}		% Largura
	
%	\imprimirautor
	
%	\hspace{0.5cm} \imprimirtitulo  / \imprimirautor. --
%	\imprimirlocal, \imprimirdata-
	
%	\hspace{0.5cm} \pageref{LastPage} p. : il. (algumas color.).\\
	
%	\hspace{0.5cm} \imprimirorientadorRotulo~\imprimirorientador\\
	
%	\hspace{0.5cm}
%	\parbox[t]{\textwidth}{\imprimirtipotrabalho~--~\imprimirinstituicao,
%	\imprimirdata.}\\
	
%	\hspace{0.5cm}

	
%	\end{minipage}
%	\end{center}
%	\hrule
%\end{fichacatalografica}
% ---


% ---
% Inserir errata (Se necessário)
% ---
%\begin{errata}
%Elemento opcional da \citeonline[4.2.1.2]{NBR14724:2011}. Exemplo:
%
%\vspace{\onelineskip}
%
%FERRIGNO, C. R. A. \textbf{Tratamento de neoplasias ósseas apendiculares com
%reimplantação de enxerto ósseo autólogo autoclavado associado ao plasma
%rico em plaquetas}: estudo crítico na cirurgia de preservação de membro em
%cães. 2011. 128 f. Tese (Livre-Docência) - Faculdade de Medicina Veterinária e
%Zootecnia, Universidade de São Paulo, São Paulo, 2011.
%
%\begin{table}[htb]
%\center
%\footnotesize
%\begin{tabular}{|p{1.4cm}|p{1cm}|p{3cm}|p{3cm}|}
%  \hline
%   \textbf{Folha} & \textbf{Linha}  & \textbf{Onde se lê}  & \textbf{Leia-se}  \\
%    \hline
%    1 & 10 & auto-conclavo & autoconclavo\\
%   \hline
%\end{tabular}
%\end{table}
%
%\end{errata}
% ---

% ---
% Inserir folha de aprovação
% ---

% Isto é um exemplo de Folha de aprovação, elemento obrigatório da NBR
% 14724/2011 (seção 4.2.1.3). Você pode utilizar este modelo até a aprovação
% do trabalho. Após isso, substitua todo o conteúdo deste arquivo por uma
% imagem da página assinada pela banca com o comando abaixo:
%
% \includepdf{folhadeaprovacao_final.pdf}
%
%\begin{folhadeaprovacao}

 %   \begin{center}
%    {\ABNTEXchapterfont\bfseries\Large\imprimirtitu%lo}
    
%    \vspace*{\fill}
    
    
%    \vspace*{\fill}
%  	{\ABNTEXchapterfont\bfseries\large\MakeUppercase{\imprimirautor}}
    
%    \vspace*{\fill}
%     \MakeLowercase{\imprimirtipodoc}
        
%    \vspace*{\fill}
%    {\ABNTEXchapterfont\bfseries\Large\imprimirtitulacao}
    
%    \vspace*{\fill}
   
      
%    \vspace*{\fill}
    
   % \normalsize\MakeUppercase{%
  %  Curso de \imprimirnivelgrau \, em %\imprimircurso \, do
 %   centro de ciências tecnológicas da
 %   universidade do estado de santa catarina.}	
	
%	\vspace*{\fill}
%	\begin{minipage}[h!]{0.35\textwidth}
%		\centering
	

%	\imprimirlocal,
%	\par 
%	\imprimirfulldata:
%	\end{minipage}    
  % 	\hfill
%	\begin{minipage}[h!]{0.62\textwidth}
 
 
   %\assinatura{\textbf{Professor} \\ Convidado 4}
%   \end{minipage}  
 %  \end{center}
%\end{folhadeaprovacao}
% ---

% ---
% Dedicatória
% ---
%\begin{dedicatoria}
   %\vspace*{\fill}
   %\centering
   %\noindent
   %\textit{ Agradeço a meus colegas e professores %por me ajudarem no meu caminho acadêmico.} %\vspace*{\fill}
%\end{dedicatoria}
% ---

% ---
% Agradecimentos
% ---
%\begin{agradecimentos}
%Gostaria de agradecer...
%
%A todos que me ajudaram em minhas dificuldades.
%\end{agradecimentos}
% ---

% ---
% Epígrafe
% ---
%\begin{epigrafe}
    %\vspace*{\fill}
	%\begin{flushright}
		%\textit{O esforço traz grandes frutos , %mais também suas responsabilidades.}
%	\end{flushright}
%\end{epigrafe}
% ---

% ---
% RESUMOS
% ---

% resumo em português
\begin{resumo}
	O objetivo deste trabalho e produzir um software para melhorar a logística de ordens de manutenção que hoje em dia e feito por meio manual trazendo inúmeros problemas e prejuízos para  a empresa. Neste documento irá ser descrito e apresentado o problema , esclarecendo os objetivos e mostrando uma solução.

 \vspace{\onelineskip}
    
 \noindent
 
\end{resumo}

% resumo em inglês
\begin{resumo}[Abstract]
 \begin{otherlanguage*}{english}
	The purpose of this work is to produce a software to improve the logistics of maintenance orders that today is done by manual means bringing numerous problems and damages to the company. In this document the problem will be described and presented, clarifying the objectives and showing a solution.

   \vspace{\onelineskip}
 
   \noindent 

 \end{otherlanguage*}
\end{resumo}

% ---
% inserir lista de ilustrações
% ---
\pdfbookmark[0]{\listfigurename}{lof}
\listoffigures*
\cleardoublepage
% ---

% ---
% inserir lista de tabelas
% ---
\pdfbookmark[0]{\listtablename}{lot}

\cleardoublepage
% ---