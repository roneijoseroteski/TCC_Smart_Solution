%% abtex2-modelo-include-comandos.tex, v-1.7.1 laurocesar
%% Copyright 2012-2013 by abnTeX2 group at http://abntex2.googlecode.com/ 
%%
%% This work may be distributed and/or modified under the
%% conditions of the LaTeX Project Public License, either version 1.3
%% of this license or (at your option) any later version.
%% The latest version of this license is in
%%   http://www.latex-project.org/lppl.txt
%% and version 1.3 or later is part of all distributions of LaTeX
%% version 2005/12/01 or later.
%%
%% This work has the LPPL maintenance status `maintained'.
%% 
%% The Current Maintainer of this work is the abnTeX2 team, led
%% by Lauro César Araujo. Further information are available on 
%% http://abntex2.googlecode.com/
%%
%% This work consists of the files abntex2-modelo-include-comandos.tex
%%

% ---
% Este capítulo, utilizado por diferentes exemplos do abnTeX2, ilustra o uso de
% comandos do abnTeX2 e de LaTeX.
% ---
 
\chapter{Introdução}



%A empresa parceira para o desenvolvimento do projeto integrador da intituição de ensino SENAI, será a Duas Rodas Alimentos,  que  possui sua matriz na cidade de  em Jaraguá do Sul, Santa Catarina  \footnote{\url{ https://www.duasrodas.com/}}. Ela utiliza como software ERP o SAP, que se trata de um software de gestão de empresas falar sobre o SAP

%Com os avanços tecnológicos se torna cada vez mais essencial que as empresas se adaptem as tendências do mercado, como a empresa parceira pa uma delas e que será abordada neste documento, é a digitalização de processos em setores de empresas da área industrial, permitindo maior eficiência nos processos e controle das informações
A empresa parceira para o desenvolvimento do projeto integrador da instituição de ensino SENAI, será a Duas Rodas Alimentos, que possui sua matriz na cidade de Jaraguá do Sul, Santa Catarina. Ela utiliza como ERP (Enterprise Resource Planning) o SAP (Second Audio Program), que se trata de um software de gestão de empresas.
Com os avanços tecnológicos acontecendo cada vez rápido, torna-se ainda mais essenciais que as empresas se adaptem as tendências do mercado, permitindo maior eficiência nos processos e controle das informações.

% ---
\section{Tema}
% ---
%Este projeto envolve a digitalização dos processos de uma empresa no setor industrial, a geração de ordens de manutenção, o controle de estoque e desenvolvimento de aplicativo para celulares. O projeto aborda um sistema de gestão de manutenção, que possa gerir as atividades realizadas pelos manutentores, com um abordagem voltada para aumentar a eficiência das atividades realizadas. 
Este projeto está focado na digitalização dos processos de manutenção no setor industrial da empresa em questão, geração de ordens de serviço, consulta de estoque e desenvolvimento de um aplicativo para smartphones. O projeto aborda um sistema de gestão de manutenção, que possa gerir as atividades realizadas pelos manutentores, com foco em aumentar a eficiência das atividades realizadas.
% ---
\section{Objetivo do Projeto}
% ---
%Este projeto tem como objetivo o desenvolvimento de um sistema que digitalize o processo de geração de ordens de manutenção bem como a manipulação desses dados para um melhor gerenciamento. Outra necessidade é com relação a integração do SAP com o Smart Solution para agilizar o processo de retorno das ordens de serviço, além disso nosso projeto tem por finalidade melhorar o gerenciamento dos dados, relatórios, consulta, controle de estoque, histórico e gráficos.
Este projeto tem como objetivo o desenvolvimento de um sistema que digitalize o processo de geração de ordens de manutenção bem como a manipulação desses dados para um melhor gerenciamento. Outra necessidade é com relação a integração do SAP com o \textit{Smart Solution} para agilizar o processo de retorno das ordens de serviço, além disso o projeto tem por finalidade melhorar o gerenciamento dos dados, relatórios, consulta, controle de estoque, histórico e gráficos.
% ---
\section{Delimitação do Problema}
% ---

A base do projeto em questão tem como foco a digitalização da geração e controle de ordens de serviço, para que o sistema se adeque as necessidades da empresa Duas Rodas, o sistema deverá conter uma aplicação Web para que os funcionários com permissão de acesso possam trabalhar, também será desenvolvido uma aplicação mobile, para tornar mais dinâmico o trabalho dos manutentores e dessa forma analisar as ordens em que deveram executar.
Com a análise dos requisitos iniciais da empresa, também será desenvolvido um sistema de estoque para o almoxarifado, visando facilitar o controle e requisição de peças conforme a demanda. Por fim, o sistema deverá ser compatível com o sistema SAP, atualmente em execução na empresa.


\section{Método de Trabalho}
%O projeto será desenvolvido em equipes de 3 a 4 pessoas, com funções específicas para a realização das diversas tarefas. Para o desenvolvimento do projeto, serão utilizados os conceitos de orientação a objetos, além de ser necessário outros conhecimentos técnicos, como o desenvolvimento para aplicativos mobile através de Ionic e Node.Js, sistemas Web com o framework Angular. O projeto será desenvolvido ao longo de cada semestre em diversos entregáveis, estes terão datas pré-definidas, que deveram ser atendidas pelos membros da equipe.
O projeto será desenvolvido em equipe, com funções específicas para a realização das diversas tarefas. Para o desenvolvimento do projeto, foram utilizados os conceitos de orientação a objetos, além de ser necessário outros conhecimentos técnicos, como o desenvolvimento para aplicativos mobile através de Ionic e Node.js, sistemas Web com framework VueJs. O projeto será desenvolvido ao longo de cada semestre em diversos entregáveis, com datas pré-definidas que deveram ser atendidas pelos membros da equipe.
% ---
\section{Organização do Trabalho}
%Este trabalho está dividido em 3 seções onde a seção 1 fala sobre o problema principal, os objetivos do projeto, delimitações do problema, a seção 2 descreve nossa solução para o problema, ferramentas utilizadas e por fim as considerações finais e futuros trabalhos.
Este trabalho está dividido em  3 seções onde a seção 1 trata  dos problemas principais, os objetivos do projeto, delimitações do problema, a seção 2 descreve a solução para o problema, ferramentas utilizadas e por fim as considerações finais e futuros trabalhos.