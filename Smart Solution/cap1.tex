
\chapter{Introdução}

{\color{red}
	
Para o desenvolvimento de projetos ligados diretamente a industria, a Faculdade Senai de Jaraguá do sul tem adotado o método de \textit{Projetos Integradores}, que é uma estratégia de ensino–aprendizagem cujo objetivo é proporcionar a interdisciplinaridade entre todos os temas/assuntos/bases abordados durante o curso de operador de computador.

O processo de realização do \textit{Projeto Integrador} fornece subsídios para a avaliação das competências relacionadas ao perfil profissional do operador de computador e seu objetivo maior é “articular teoria e prática” mediante o contato do aluno com diversos contextos do mundo do trabalho com as industrias da região \cite{magalhaes2019uso}.  

A empresa parceira para o desenvolvimento do \textit{Projeto Integrador} da foi a Duas Rodas Alimentos,  que  possui sua matriz na cidade de  em Jaraguá do Sul, Santa Catarina  \footnote{\url{ https://www.duasrodas.com/}}.  Ela utiliza como ERP (Enterprise Resource Planning) o SAP (Second Audio Program), que trata-se de um software de gestão de empresas.
Com os avanços tecnológicos acontecendo cada vez rápido, torna-se ainda mais essenciais que as empresas se adaptem as tendências do mercado, permitindo maior eficiência nos processos e controle das informações.

}
% 


\section{Tema}
% ---
%Este projeto envolve a digitalização dos processos de uma empresa no setor industrial, a geração de ordens de manutenção, o controle de estoque e desenvolvimento de aplicativo para celulares. O projeto aborda um sistema de gestão de manutenção, que possa gerir as atividades realizadas pelos manutentores, com um abordagem voltada para aumentar a eficiência das atividades realizadas. 
Este projeto está focado na digitalização dos processos de manutenção no setor industrial da empresa em questão, geração de ordens de serviço, consulta de estoque e desenvolvimento de um aplicativo para smartphones. O projeto aborda um sistema de gestão de manutenção, que possa gerir as atividades realizadas pelos manutentores, com foco em aumentar a eficiência das atividades realizadas.
{\color{red} Melhorar o Tema de vcs, pelo menos umas 10 linhas bem detalhadas}
% ---
\section{Objetivo do Projeto}
% ---
%Este projeto tem como objetivo o desenvolvimento de um sistema que digitalize o processo de geração de ordens de manutenção bem como a manipulação desses dados para um melhor gerenciamento. Outra necessidade é com relação a integração do SAP com o Smart Solution para agilizar o processo de retorno das ordens de serviço, além disso nosso projeto tem por finalidade melhorar o gerenciamento dos dados, relatórios, consulta, controle de estoque, histórico e gráficos.
Este projeto tem como objetivo o desenvolvimento de um sistema que digitalize o processo de geração de ordens de manutenção bem como a manipulação desses dados para um melhor gerenciamento. Outra necessidade é com relação a integração do SAP com o \textit{Smart Solution} para agilizar o processo de retorno das ordens de serviço, além disso o projeto tem por finalidade melhorar o gerenciamento dos dados, relatórios, consulta, controle de estoque, histórico e gráficos e por consequência, reduzir o consumo de papel.

{\color{red} ok}
% ---
\section{Delimitação do Problema}
% ---

A base do projeto em questão, tem como foco a digitalização da geração e controle de ordens de serviço, para que o sistema se adeque as necessidades da empresa Duas Rodas, o sistema contem uma aplicação Web para que os funcionários com permissão de acesso possam trabalhar, também foi desenvolvido uma aplicação mobile, para tornar mais dinâmico o trabalho dos manutentores e dessa forma analisar as ordens em que foi executado.
Com a análise dos requisitos iniciais da empresa, também foi desenvolvido um sistema de estoque para o almoxarifado, visando facilitar o controle e requisição de peças conforme a demanda. Por fim, o sistema foi estruturado para ser compatível com o sistema SAP, atualmente em execução na empresa.

{\color{red} Como o projeto ja foi desenvolvido boa parte dele cuidar com os verbos SERÁ desenvolvido  por FOI desenvolvido, pois vcs ja executaram}


\section{Método de Trabalho}
%O projeto será desenvolvido em equipes de 3 a 4 pessoas, com funções específicas para a realização das diversas tarefas. Para o desenvolvimento do projeto, serão utilizados os conceitos de orientação a objetos, além de ser necessário outros conhecimentos técnicos, como o desenvolvimento para aplicativos mobile através de Ionic e Node.Js, sistemas Web com o framework Angular. O projeto será desenvolvido ao longo de cada semestre em diversos entregáveis, estes terão datas pré-definidas, que deveram ser atendidas pelos membros da equipe.
O projeto foi desenvolvido em equipe, com funções específicas para a realização das diversas tarefas. Para o desenvolvimento do projeto, foram utilizados os conceitos de orientação a objetos, além de ser necessário outros conhecimentos técnicos, como o desenvolvimento para aplicativos mobile através de Ionic e Node.js, sistemas Web com framework VueJs. O projeto foi desenvolvido ao longo de cada semestre em diversos entregáveis, com datas pré-definidas que deveram ser atendidas pelos membros da equipe.
{\color{red} Cuidar com os verbos, melhorar a escrita, e onde vcs chamaram as ferramantas utilizadas, linkar  o repositório de onde baixar, site etc.}

% ---
\section{Organização do Trabalho}
%Este trabalho está dividido em 3 seções onde a seção 1 fala sobre o problema principal, os objetivos do projeto, delimitações do problema, a seção 2 descreve nossa solução para o problema, ferramentas utilizadas e por fim as considerações finais e futuros trabalhos.
Este trabalho está dividido em  3 seções onde a seção 1 trata  dos problemas principais, os objetivos do projeto, delimitações do problema, a seção 2 descreve a solução para o problema, ferramentas utilizadas e por fim as considerações finais e futuros trabalhos.

{\color{red} Vcs tem quantas seções no projeto??? descrever as reais}