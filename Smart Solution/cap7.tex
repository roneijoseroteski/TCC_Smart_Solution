\chapter{Utilização de Gamificação para o Sistema }
% ---
Nesse capítulo será evidenciado formas que estimulara os usuários a produzir mais e hospedagem para que o Smart Solution seja colocado fora do ambiente de desenvolvimento. {\color{red} Melhorar a introdução}
% ---
%\section{Login social}
% ---
%Apreendemos a usar API de terceiros para registrar apenas as informações desejadas em nosso sistema, para que haja maior segurança para nosso cliente.
% ---
\section{Gamificação}
% ---

{\color{red} Adicionar referencias(2) sobre gamificação }

A gamificação está muito presente em diversos sistemas e plataformas como uma forma de entreter o usuário utilizando mecânicas e dinâmicas de jogos eletrônicos para engajar as pessoas, com esse conceito, inicialmente foi aplicado ao projeto a opção de classificação das ações realizadas em uma ordem de serviço feita, dessa forma o operador ou responsável pode avaliar o trabalho realizado pelo manutentor trazendo assim feedbacks positivos ou negativos, possibilitando um maior controle de qualidade e se necessário discutir melhorias nos processos.

%referencia a gamificação


% ---
%\section{Marketing}
% ---
%Apreendemos marketing em sala de aula para a ter ma visão ampla da concorrência,de como fazer nosso sistema se destacar, segmentar mercado, a estipular preços,marca, logotipo para nosso produto.
% ---
\section{Hospedagem}
% ---

{\color{red} Adicionar referencias(2) sobre hospedagem }

Como forma de hospedagem uma das tecnologias estudas foi a da IBM Cloud que permite um ambiente de implantação seguro e rápido, pois o mesmo é conectado com o GitHub, dessa forma todo o código é gerenciado pela IBM Cloud na medida em que o projeto é lançado, agilizando o processo de implementação e possibilitando as atualizações do sistema de forma ágil e segura.
% ---