%\chapter{Implantação }
\chapter{Conclusões e Trabalhos Futuros}

Com o desenvolvimento do projeto Smart Solution pode-se perceber a importância de um sistema WEB e Mobile para o mercado industrial. Para tal fato, o projeto em questão teve inicialmente como foco principal o desenvolvimento da parte mobile, onde foi implementado sua interface, aplicado as regras de negócio e concluída a API de conexão com o banco de dados. Sendo assim o projeto integrador encontra-se parcialmente funcional, podendo sofrer melhorias no decorrer de todo o processo de desenvolvimento.


{\color{red} Modificar para o projeto atual agora, ja pensando que estão finalizando ele

Para as próximas etapas será melhorado o sistema de segurança da aplicação com novas implementações e adaptações, melhorando suas funcionalidades. Também será desenvolvimento a aplicação WEB visando a melhoria do gerenciamento das ordens de serviço e a possibilidade de novos cadastros caso haja impossibilidades de comunicação com o SAP.

Para finalizar os diferenciais do Smart Solution para soluções já implementadas, foi proposta a inclusão de um módulo de controle de estoque. Este módulo permite que haja um gerenciamento do estoque do almoxarifado e consequentemente facilitando as solicitações de novos equipamentos ou peças.

Sendo assim, dentre as soluções propostas, o Smart Solution será o sistema que melhor se adaptará as necessidades da empresa Duas Rodas, contando com a versão Mobile e WEB proporcionando maior liberdade aos colaboradores.

}








